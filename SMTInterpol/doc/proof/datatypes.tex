\documentclass[a4paper]{article}
\usepackage{xspace}
\usepackage{relsize}
\usepackage{xcolor}
\usepackage{amsmath}
\usepackage{pxfonts}
\usepackage{mathpartir}
\newcommand\mT{\mathcal{T}}
\newcommand\T{$\mT$\xspace}
\newcommand\Tp{$\mT'$\xspace}
\newcommand\mtp{\models_{\mT'}}
\newcommand\syms{\mathop{\mathit{syms}}}
\newcommand\si{SMTInterpol\xspace}
\newcommand\sversion{2.0}
\newcommand\siv{\si~\sversion}
\newcommand\gen[1]{\mathop{gen}\nolimits_{#1}}
\newcommand\dom{\mathop{dom}}
\newcommand\todo[1]{\textcolor{red}{TODO: #1}}
\newcommand\seqcomp{\circledwedge}
\newcommand\choicecomp{\circledvee}

\newcommand\true{\mathbf{true}}
\newcommand\false{\mathbf{false}}
\newcommand\isC{\mathit{isC}}

\title{Datatype Proof Rules}
\author{Jochen Hoenicke \and Moritz Mohr}
\date{2021/01/31}
\begin{document}
\maketitle
\section{Proof Rules for Theory of Data Types}

\paragraph{Rule 1} (dt-project)

\[
\frac{C_i(v_1,\dots,v_n) \sim u, s_{ij}(u) \in V}{s_{ij}(u) \sim v_j}
\]

This rule ensures that $s_{ij}(C_i(v_1,\dots,v_n)) = v_j$.  It is
instantiated whenever there is a select term $s_{ij}(u)$ among the
E-graph terms $V$ whose argument $u$ equals some matching constructor
term.


\paragraph{Rule 2a} (dt-tester)

\[
\frac{C_i(v_1,\dots,v_n) \sim u, \isC_i(u)\in V}{\isC_i(u) \sim \true}
\]

This rule ensures that $\isC_i(C_i(v_1,\dots,v_n)) = \true$.  It is
instantiated whenever there is a test term $\isC_i(u)$ among the
E-graph terms $V$ whose argument $u$ equals some matching constructor
term.


\paragraph{Rule 2b} (dt-tester)

\[
\frac{C_i(v_1,\dots,v_n) \sim u, \isC_j(u)\in V}{\isC_j(u) \sim \false}
\]

Similarly, this rule ensures that $\isC_j(C_i(v_1,\dots,v_n)) = \true$.  It is
instantiated whenever there is a test term $\isC_j(u)$ among the
E-graph terms $V$ whose argument $u$ equals some constructor
term with a different constructor.


\paragraph{Rule 3} (dt-constructor)

\[
\frac{\isC_i(u) \sim \true, s_{i1}(u_1) \in V,u \sim u_1,\dots,s_{in}(u_n)\in V,u \sim u_n}{u \sim C_i(s_{i1}(u_1),\dots, s_{in}(u_n)}
\]

This rule is instantiated when there is a test term $\isC_i(u)$ that
is equal to true in the E-graph and for each corresponding selector there
is a term $s_{ij}(u_j)$ where $u_j$ equals $u$ in the E-graph.
Then the new term $C_i(s_{i1}(u_1),\dots,s_{in}(u_n))$ is created and set
equal to $u$.

\paragraph{Rule 4}
\[
\frac{u \in V, |\tau_{s_{ij}}| < \infty,
  |\tau_{ik}| < \infty|\text{ or }s_{ik}(u_k)\in V, u\sim u_k 
  \text{ for }k\neq j}
     {\text{add $s_{ij}(u)$ to $V$}}
\]

This rule creates the $s_{ij}(u)$ term if it currently does not
exists, but may be important because the range of the selector is
finite.  It is only important if the other selectors exists or are
also finite.  In this case it may be necessary to enumerate all
possible values for $s_{ij}(u)$ to find a conflict in each case.  By
creating the term, the corresponding theory for the finite domain is
forced to choose a concrete value for it.  Note that the term is only
added if no equivalent term $s_{ij}(u_j)$ with $u\sim u_j$ already
exists.

\paragraph{Rule 5}
\[
\frac{u \in V, s_{i1}(u_1)\in V,u\sim u_1,\dots, s_{in}(u_n)\in V, u\sim u_n}{\text{add $\isC_i(u)$ to $V$}}
\]

Similar to the previous rule, this creates the $\isC_i(u)$ term if all
selectors exists.  If $\isC_i(u)$ is then given the value $\true$,
Rule~3 triggers.


\paragraph{Rule 6} (dt-cases)
\[
\frac{\isC_1(u_1)\sim \false,\dots, \isC_n(u_n)\sim \false, u_1\sim\dots\sim u_n}{\bot}
\]
This rule reports a conflict if all test functions are currently $\false$.

\paragraph{Rule 7} (dt-cycle)

TODO

\paragraph{Rule 8} (dt-injective)
\[
\frac{C_i(a_1,\dots,a_n) \sim C_i(b_1,\dots,b_n)}{a_1 \sim b_1,\dots, a_n\sim b_n}
\]

This rule ensures that constructors are injective and propagates the
equality between their arguments.


(dt-disjoint)
\[
\frac{C_i(a_1,\dots,a_n) \sim C_j(b_1,\dots,b_m)}{\bot}
\]

This rule reports a conflict if two different constructors are equal in
the E-graph.

\paragraph{Rule 9} (dt-unique)
\[
\frac{\isC_i(u_1) \sim \true, \isC_j(u_2)\sim \true, u_1\sim u_2)}{\bot}
\]

This rule reports a conflict if two different tester functions are both
true in the E-graph.


\end{document}
